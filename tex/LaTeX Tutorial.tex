\documentclass[11pt, oneside]{article}   	% use "amsart" instead of "article" for AMSLaTeX format
\usepackage{kat} % you can change the name for this, look in kat.sty

%%%%% 
\begin{document} \pagestyle{empty}

\begin{flushright}
Your name here
\\
Your student ID or \# here
\end{flushright}

\vspace{12pt}

\begin{center}
\textbf{\Large{Course Code 123: Assignment $\mathbf{n}$}}
\end{center}

\vspace{24pt}

%%%%%%%%%%%%%%%%%%%%%%%%%%%%%%%%%%
%%delete-me

This .tex \LaTeX \ file is the learning version, prepared by Kat Matheson.

Any questions should be directed to \texttt{g3.matheson@gmail.com}

$\ $

If you lost the template, simply delete everything below the comment \emph{\%\%delete-me} and save your own.

Make sure to keep the $\backslash$end\{document\} line, though or it might go on forever.

\newpage % let's go to the next page

\tableofcontents % this will show \section{} and \subsection{}
		 % you can use \section*{} if you don't want them numbered

\newpage

%%%%%%%%%%%%%%%%%%%%%%%%%%%%%%%%%%%%%%%%%%%%%%%%%%
%%%%%%%%%%%%%%%%%%%% LaTeX %%%%%%%%%%%%%%%%%%%%%%%
%%%%%%%%%%%%%%%%%%%%%%%%%%%%%%%%%%%%%%%%%%%%%%%%%%
\section{\LaTeX} \noindent

%%%% LaTeX basics
\LaTeX \ has odd spacing.

if you write a line
and continue on the next one, you might notice a problem

if you write a line and something else \\
and continue on the next one, you might notice it's fixed, sort of? \\
what if...
hmmmmm

now?








how about now?











where are my spaces?

i guess i only need one...


%% vertical spacing
\vspace{60pt} 

i could do this though

%% centering
\begin{center}
i can write in the middle
\end{center}

\begin{flushright}
or over here to write my name and student ID in the top right corner
\end{flushright}

% \textbf{string s}
this is normal text, \textbf{but this is bolded.}

% \textit{} or \emph{}
this is normal text, \textit{this is italicized}, and \emph{this is emphasized}


\vspace{36pt}

% \texttt{} 
this is normal text, \texttt{and this is cool math-ish pc-looking text}

% sizing
{\large this is large} 
{\Large this is Large}
{\small or you can be small}
{\tiny or even tiny}

\vspace{1in}


%% Spacing
%%%%%%%%%%%%%%%%%%%%%%%
%%%%%% horizontal spacing
% in pt (points/pixels? not sure tbh)
\hspace{200pt} over here


% or inches:
\hspace{2in} weeeeeeeeeeeeeeee

% causes the error below of "Overfull \hbox":
\hspace{7in} weeeeeeeeeeeeeeee

% for some reason this one isn't an error...
\hspace{-1.5in} weeeeeeeeeeeeeeee


	% the re-sizing is based on the math engine, but it can
	% also tell what height your letters are.

\newpage



%%%%%%%%%%%%%%%%%%%%%%%%%%%%%%%%%%%%%%%%%%%%%%%%%%
%%%%%%%%%%%%%%%%%%%% Math %%%%%%%%%%%%%%%%%%%%%%%%
%%%%%%%%%%%%%%%%%%%%%%%%%%%%%%%%%%%%%%%%%%%%%%%%%%
\section{Math} \noindent

% math environment is inside of ${...}$
a,b,c,d,e,f,g,h,i,j,k,l,m,n,o,p,q,r,s,t,u,v,w,x,y,z

1,2,3,4,5,6,7,8,9,0

$a,b,c,d,e,f,g,h,i,j,k,l,m,n,o,p,q,r,s,t,u,v,w,x,y,z$

$1,2,3,4,5,6,7,8,9,0$ % notice spacing is automatic

%% math environment can also be centered/emphasized with $$
\[ a,b,c = 1,2,3 = \sum_{i=1}^j \int_0^{\infty} e^{ix} \ dx \]

$\ $ % math spacer, just takes up an (almost) empty line
		 % for when you're too lazy to type \vspace{12pt}

% greek letters for variables
\texttt{greek letters for variables}

$\alpha, \beta, \gamma, \sigma, \theta, \epsilon, \varepsilon$

$\ $

% math symbols
\texttt{math symbols}

$\sum_{i=1}^n$, $\prod_{i=1}^n$, $\bigcap_{i=1}^n$, $\bigcup_{i=1}^n$

$a+b$, $a-b$, $a \pm b$

$\frac{a}{b}, a/b$

$a > b, a \geq b, a < b, a \leq b$

$a \ne b, a \sim b, a \approx b, a \simeq b$

$a \cdot b, a \times b, a * b$

% pay attention to subscripts!
% they must be in {} if they're not just 1 character
$a^b, a^{2b}, a^2b$

$a_b, a_{2b}, a_2b$ 

$A \cup B, A \cap B, A \setminus B$

$A \subset B, A \subseteq B, A \supset B, A \supseteq B$

$\in, \notin$

$\forall$, $\exists$, $\implies$, $\impliedby$, $\iff$

$a \land b, a \lor b$

$\bP(a+b) < \bE(c \pm d)$

$x \in \bN, y \in \bR, z \notin Z, \alpha \in \bQ\setminus(\bN \cap \bQc)$

and so on $\cdots$

and so on $\ldots$

$\ $

% math functions
\texttt{math functions}

$\sin(x), \cos^2(x), \tan(\theta)$

$e^{x_i^2}, \log_2(x)$

\[ f^{\prime\prime} = f^{(2)} = \frac{d^2 f}{dx^2} \ne \frac{\partial^2 f}{\partial x \partial y} \]

$\ $

% math auto-sizing brackets
\texttt{math auto-sizing brackets}

$\bp{2+2}$

$\bb{\bp{2+\frac{1}{2}}}$

$\bc{\bp{\bp{2 + 2}} > \bigcap_{i=1}^n \frac{\sum_{i=1}^n \bp{i +1}^2}{\theta_1}}$

\[ \bp{1 + \bigcap_{i=1}^n \frac{\sum_{i=1}^n \bp{i +1}^2}{\theta_1}} \]

\newpage

% math comments
\texttt{math comments}
\[ a = b \text{ because science} \]
\[ and d = e \ \text{because} \]
\[ b = c \hspace{2in} \text{according to my mom} \]

% align environment
\begin{align*}
x &= y \\
&\sim \theta - 3 \ \text{and some other text} \\
&= z + 1 \quad \text{and some other text} \\
&< z + 3 \qquad \text{and some other text}
\end{align*}

% multiple-align environment
\begin{align*}
x &= 2 & i &= 1 \\
x &= 3 & i &= 2
\end{align*}

% matrices
\texttt{matrices}
\[
\begin{bmatrix}
    0,1,2 \\
    3,4,5 \\
    6,7,8 
\end{bmatrix}
\begin{pmatrix}
    0,1,2 \\
    3,4,5 \\
    6,7,8
    \end{pmatrix}
\]

% cases
\texttt{cases}
\[
f(x) = \begin{cases}
    \frac{1}{2} &\text{if } x > 0 \\
    \theta_0    &\text{if } x = 0 \\
    0           &\text{otherwise}
\end{cases}
\]

\newpage
\begin{center}
\texttt{Examples}

\tiny{common mistakes were made on purpose, to show what they do, and why certain things are useful}
\end{center}
\begin{enumerate}[(1)]
    \item $\sum_{i=0,i\ne k}^{n-1} x_i^2$
    % the same function but in \[ \] instead of $ $, indices go to top and bottom
    \item \[\sum_{i=0,i\ne k}^{n-1} x_i^2\]
    \item $\sum_{i=1}^n \frac{x_i+y_i}{{2^i}^{i-1}}$
    \item $\prod_{i=1}^n \bp{\frac{x_i+y_i}{2^i}}^{i-1}$
    \item $\frac{\sum_{i=1}^n x_i}{n}$
    \item $\frac{\int_0^1 \frac{a}{x^{-2}} \ dx}{2}$
\end{enumerate}




\newpage

%%%% to be continued...

%%%%%%%%%%%%%%%%%%%%%%%%%%%%%%%%%%%%%%%%%%%%%%%%%%
%%%%%%%%%%%%%%%%% Algorithms %%%%%%%%%%%%%%%%%%%%%
%%%%%%%%%%%%%%%%%%%%%%%%%%%%%%%%%%%%%%%%%%%%%%%%%%
\section{Algorithms} \noindent

%%%%%%%%%%%%%%%%%%%%%%%%%%%%%%%%%%%%%%%%%%%%%%%%%%
%%%%%%%%%%%%%%%%%%% Trees %%%%%%%%%%%%%%%%%%%%%%%%
%%%%%%%%%%%%%%%%%%%%%%%%%%%%%%%%%%%%%%%%%%%%%%%%%%
\section{Trees} \noindent

\hyperlink{https://tikz.dev/tikz-trees}{more on drawing trees (here)} (this is a hyperlink, it will open the default browser from a pdf click)

see the .tex file for how to draw the tree

\begin{center}
    \begin{tikzpicture}                     % 'levels' = depth
    [level distance = 10mm,                     % distance between parents and children
     level 1/.style={sibling distance=30mm},    % adjust as necessary to make space
     level 2/.style={sibling distance=15mm}]    % add more levels if needed
        \node {A}
            child 
            {node {B}
                    child {node {D}}
                    child {node {E}}}
            child 
            {node {C}};
    \end{tikzpicture}
    \end{center}


\end{document}
